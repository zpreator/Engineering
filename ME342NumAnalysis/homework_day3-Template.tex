\documentclass[11pt]{article}
\setlength{\baselineskip}{13pt}
\setlength{\parskip}{10pt plus 1pt minus 1pt}
\setlength{\parindent}{0em}
\usepackage[margin=1.0in]{geometry}
\pagestyle{empty}

%%%%%%%%%%%%%%%%   Font choice   %%%%%%%%%%%%%%%%%%%%%%
% Times-Roman
\renewcommand\familydefault{ptm}
\DeclareMathAlphabet{\mathmyrm}{OT1}{ptm}{m}{n}
\DeclareMathAlphabet{\mathmybf}{OT1}{ptm}{bx}{n}
\SetMathAlphabet{\mathmyrm}{bold}{OT1}{ptm}{bx}{n}
% or the new way (but I don't like the Times math -- rho is too big
%\usepackage{mathptmx}
%%%%%%%%%%%%%%%   Font choice (end)    %%%%%%%%%%%%%%%%%%%%%

%%%%%%%%%%%%%%%%   Graphics   %%%%%%%%%%%%%%%%%%%%%%%%%
\usepackage{graphicx}
%\includegraphics[width=1in, angle=270]{w97/t1plane.pstex}
\usepackage{subfig}
%%%%%%%%%%%%%%%   Graphics (end)   %%%%%%%%%%%%%%%%%%%%%%%

%%%%%%%%%%%%%%%%   Hyperref    %%%%%%%%%%%%%%%%%%%%%%%%%
\usepackage[colorlinks=true]{hyperref}
\hypersetup{colorlinks=true, linkcolor=blue, filecolor=blue, pagecolor=blue, urlcolor=blue}
%%%%%%%%%%%%%%    Hyperref (end)  %%%%%%%%%%%%%%%%%%%%%%%%%

%%%%%%%%%%%%%%&    Other     %%%%%%%%%%%%%%%%%%%%%%%%%%%%
\usepackage{color}
\usepackage{latexsym}
\usepackage{amsmath,amssymb,amsfonts,textcomp}
\usepackage{fixltx2e}
\usepackage{calc}
\usepackage{longtable}
\usepackage{booktabs} %for nice looking tables

%set up a few equation variables substitutions
\newcommand{\textnum}[1]{\ensuremath{\mathmyrm{#1}}}
\newcommand{\dgr}{\boldmath\ensuremath{{}^\circ}\unboldmath}
\newcommand{\vlm}{V \!\!\!\!\!\!\: \rule[0.07in]{.095in}{.0025in}}
%%%%%%%%%%%%%%%%%%%%%%    Other (end)  %%%%%%%%%%%%%%%%%%%%%%%%%%%%


\begin{document}

\begin{center}
  \large Assignment Name \\
Student Name \\
Date...
\end{center}

\textbf{Simple way of adding a section title}

This is how you add an itemized list:
\begin{itemize}
\item 1st item
\item another item
\item and another ...
\end{itemize}

If you want to talk about a specific python function, like \texttt{np.max} you can make it look like computer text this way.

This is how we add a numbered equation
\begin{equation}
\frac{dy}{dt} = y \sin^3(t)
\label{eq:name}
\end{equation}
Be sure to define the variables like $y$ and $t$.

If you want an unnumbered equation, it is a bit easier to add it this way
$$ y = \int_0^{t_{final}} \rho V^2 dA $$

To provide the link to a website you can either show the full URL, like \url{http://www.byui.edu}, or you can use a hyperlink, such as \href{http://www.byui.edu}{BYUI home}. You can also link to other parts of the text: using a link to \hyperref[eq:name]{Equation 1}.

Here is code for a nice looking table (see Table~\ref{tab:nice})

\begin{table}[htbp]
\caption{Put the cation text here...}
\label{tab:nice}
\begin{center}
\begin{tabular}{ccccccc} \toprule
~ & \multicolumn{2}{c}{Bisection Method} & \multicolumn{2}{c}{Secant Method} & \multicolumn{2}{c}{Ridder's Method}  \\ 
\cmidrule(lr){2-3} \cmidrule(lr){4-5} \cmidrule(lr){6-7}
 Iteration & Root & Error & Root & Error & Root & Error  \\ \midrule 
 1 	& 0.9000 & 3.33E-1 &  0.7295 & 6.45E-1 & 1.1186 & 4.64E-01 \\
 3 	& 0.9750 & 7.69E-2 &  0.9191 & 5.64E-2 & 0.9477 & 9.13E-3 \\
 5 	& 0.9562 & 1.96E-2 &...&...&...&... \\ 
 7 	& 0.9516 & 4.93E-3 &...&...&...&... \\
 10 & 0.9475 & 6.18E-4 &...&...&...&... \\
 15 & 0.9480 & 1.93E-5 &...&...&...&... \\
 20 & 0.9480 & 6.04E-7 &...&...&...&... \\ \bottomrule
\end{tabular}
\end{center}
\end{table}

\newpage

To add a figure, you use this code, and then you can reference it with Figure~\ref{fig:converging}.

\begin{figure}[htbp]
\begin{center}
\includegraphics[width=4in]{convergingSolution}
\caption{Don't forget to put in a good caption. Captions should be about two complete sentences, and explain what the figure or table is, and why it was important to include in the document.}
\label{fig:converging}
\end{center}
\end{figure}

If you have lots of related figures that you would like to display together, you can use the \texttt{subfig} package. You can reference the full figure like this (Figure~\ref{fig:withSubfigs})

\begin{figure}[htbp]
\begin{center}
	\subfloat[Caption for subfig (a)]{\includegraphics[width=2.5in]{convergingSolution}} \qquad 
       	\subfloat[Caption for subfig (b)]{\includegraphics[width=2.5in]{convergingSolution}} \\
	\subfloat[Caption for subfig (c)]{\includegraphics[width=2.5in]{convergingSolution}} \qquad
       	\subfloat[Caption for subfig (d)]{\includegraphics[width=2.5in]{convergingSolution}}	
\caption{This would be the caption for the whole figure.}
\label{fig:withSubfigs}
\end{center}
\end{figure}


\end{document}



